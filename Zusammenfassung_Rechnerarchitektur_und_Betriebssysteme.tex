\documentclass[12pt,a4paper]{article}

\usepackage[utf8]{inputenc}
\usepackage[ngerman]{babel}
\usepackage[T1]{fontenc}
\usepackage{amsmath}
\usepackage{amsfonts}
\usepackage{amssymb}
\usepackage{graphicx}
\usepackage[left=2cm,right=2cm,top=2cm,bottom=2cm]{geometry}
\usepackage{multicol}
\usepackage{booktabs}
\usepackage[hidelinks]{hyperref}
\usepackage{tikz}
\usepackage{pgfplots}
\usepackage{blindtext}
\usepackage{array}
\usepackage{multirow}
\usepackage{bigdelim}
\usepackage{colortbl}
\usepackage{fancyhdr} 
\usepackage{tabularx}
\usepackage{pgfplots}
\usepackage{xcolor}
\usepackage{color}
\usetikzlibrary{decorations.text}
\usetikzlibrary{tikzmark}
\pagestyle{fancy} 
	\fancyhf{} 
	\fancyhead[L]{\includegraphics[scale=0.05]{Bilder/dhbw.png}} 
	\fancyhead[C]{\slshape Rechnerarchitektur und Betriebssysteme} 
	\fancyhead[R]{\slshape LaTeX Version}

\usepackage{helvet}
\renewcommand{\familydefault}{\sfdefault}

\newcolumntype{Z}{>{\centering\let\newline\\\arraybackslash\hspace{0pt}}X}
\author{\slshape Robin Rausch, Florian Maslowski, Ozan Akzebe}
\title{Rechnerarchitektur und Betriebssysteme}
\date{\slshape \today}
\begin{document}
\maketitle
\tableofcontents
\newpage
\section{Rechnerarchitektur Zusammenfassung alles hier rein}
SAMPLE TEXT 
\section{Betriebssysteme Grundlagen}
\subsection{Arten von Betriebssystemen}

\subsection{Architekturen von Betriebssystemen}
\subsubsection{Monolithisches System}

\subsubsection{Geschichtetes System}

\subsubsection{Mirkoarchitektur}

\subsubsection{Modulares System}

\subsection{Paging}
Verfahren

\subsection{Swapping}

\subsection{MMU}
Die Memory Management Unit \dots

\subsection{Prozesse}
\subsubsection{Prozesszustände}

\end{document}